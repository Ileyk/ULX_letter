%                                                                 aa.dem
% AA vers. 8.2, LaTeX class for Astronomy & Astrophysics
% demonstration file
%                                                       (c) EDP Sciences
%-----------------------------------------------------------------------
%
%\documentclass[referee]{aa} % for a referee version
%\documentclass[onecolumn]{aa} % for a paper on 1 column  
%\documentclass[longauth]{aa} % for the long lists of affiliations 
%\documentclass[rnote]{aa} % for the research notes
%\documentclass[letter]{aa} % for the letters 
%\documentclass[bibyear]{aa} % if the references are not structured 
% according to the author-year natbib style

%
\documentclass[letter]{aa}  

%
\usepackage{graphicx}
%%%%%%%%%%%%%%%%%%%%%%%%%%%%%%%%%%%%%%%%
\usepackage{txfonts}
%%%%%%%%%%%%%%%%%%%%%%%%%%%%%%%%%%%%%%%%
\usepackage{tabularx}
\usepackage{amsfonts}
\usepackage{bbold}
\usepackage{color}
\usepackage{transparent}
\usepackage{hyperref}
\usepackage{transparent}
\usepackage{rotating}
\usepackage{caption}
\usepackage{subcaption}

% Only include extra packages if you really need them. Common packages are:
\usepackage{graphicx}	% Including figure files
\usepackage{amsmath}	% Advanced maths commands
\usepackage{amssymb}	% Extra maths symbols
\usepackage{tablefootnote}
\usepackage[flushleft]{threeparttable}
\usepackage{dblfloatfix} 

\usepackage{natbib}
\bibpunct{(}{)}{;}{a}{}{,} 

\newcommand{\sgx}{SgXB\xspace}
\newcommand{\ulx}{ULX\xspace}
\newcommand{\sfxt}{SFXT}
\newcommand{\sg}{Sg\xspace}
\newcommand{\co}{CO\xspace}
\newcommand*{\hmxb}{HMXB\@\xspace}
\newcommand*{\lmxb}{LMXB\@\xspace}
\newcommand*{\rlof}{RLOF\@\xspace}
\newcommand*{\ns}{NS\@\xspace}
\newcommand*{\bh}{BH\@\xspace}
\newcommand*{\eg}{e.g.\@\xspace}
\newcommand*{\ie}{i.e.\@\xspace}
\newcommand*{\aka}{a.k.a. \@\xspace}
\newcommand*\diff{\mathop{}\!\mathrm{d}}
\newcommand{\mystar}{{\fontfamily{lmr}\selectfont$\star$}}
\newcommand*{\msun}{$M_{\odot}$\@\xspace}

%\usepackage[options]{hyperref}
% To add links in your PDF file, use the package "hyperref"
% with options according to your LaTeX or PDFLaTeX drivers.
%
\begin{document} 


   \title{Mass transfer via wind-RLOF in Supergiant X-ray binaries}

   \subtitle{A possible mechanism for Ultraluminous X-ray sources}

   \author{I. El Mellah
          \inst{1}
          \and
          J. O. Sundqvist
          \inst{2}
          \and
          R. Keppens
          \inst{1}
          }

   \institute{Centre for mathematical Plasma Astrophysics, 
   			 Department of Mathematics, KU Leuven, 
   			 Celestijnenlaan 200B, B-3001 Leuven, Belgium\\
              \email{ileyk.elmellah@kuleuven.be}
         \and
             KU Leuven, Instituut voor Sterrenkunde, 
             Celestijnenlaan 200D, B-3001 Leuven, Belgium
             }

   \date{Received ...; accepted ...}

% \abstract{}{}{}{}{} 
% 5 {} token are mandatory
 
  \abstract{Ultra-luminous X-ray sources (\ulx) have such high X-ray luminosities that they were long thought to be accreting intermediate mass black holes. Yet, some have been shown to display periodic modulations and coherent pulsations, suggestive of a neutron star in orbit around a companion and accreting at super-Eddington rates. The question of the mass transfer mechanism suitable to feed the accretor at such high rates remains open. In this letter, we propose that Supergiant X-ray binaries (\sgx) could undergo a \ulx phase when the slow line-driven wind from the hot donor star is highly beamed towards the compact accretor. Since the star does not fill its Roche lobe and that a significant fraction of the stellar wind still escapes the system, this mass transfer mechanism known as "wind - Roche lobe overflow" can remain stable even for large mass ratios. Based on line-acceleration profiles derived from spectral observations and modeling of the stellar wind, we perform three-dimensional ballistic simulations to evaluate the fraction of the wind captured by the compact object. We identify realistic orbital and stellar conditions for a \sgx to be the stage of mass transfer rates matching the expectations for \ulx and show that the transition from \sgx to \ulx luminosity levels is progressive. These results prove that high stellar Roche lobe filling factors are not necessary to funnel large quantities of material into the Roche lobe of the accretor. Slow and dense winds such as the ones emitted by the Wolf-Rayet star in M101 ULX-1 or even the cold Red Supergiant in P13 ULX-1 are enough to lead to a highly beamed wind and a significantly enhanced mass transfer rate.}

   \keywords{XXX accretion, accretion discs -- X-rays: binaries -- stars: neutron, supergiants, winds, outflows -- methods: numerical}

   \maketitle
%
%________________________________________________________________

XXX

notwithstanding
hitherto

XXX

\section{Introduction}


Ultra-luminous X-ray sources are point sources with luminosities in excess of 10$^{39}$erg$\cdot$s$^{-1}$. This X-ray luminosity threshold corresponds approximately to the Eddington luminosity of a 10 solar masses black hole (\bh), the limit above which isotropic accretion is thought to be self-regulated by the radiative field it produces \citep{Rappaport2005}. They are found off-nuclear in galaxies within a couple of 10Mpc, ruling against a supermassive black hole origin.

What is it? Webb. Review by Kaaret2017. Draw the line at 1E39erg/s (~Eddington limit for a 10 solar masses BH). Plurality of origins and at this point, we can not discard that some might be the long awaited IMBH as suggested by \cite{Colbert1999}. Actually, we know that some accretors have masses too high for a \ns : IC10 X-1 \citep{Prestwich2007,Silverman2008}. Although another scenario is investigated in this letter.

Regardless of the mechanism which explains the super-Eddington accretion rates in the immediate vicinity of the accretor, which source of matter? Likely binary systems (natural extension of the HMXB LX distribution) where a donor star plays the role of a reservoir tapped by the orbiting compact object. In P13 for instance, high mass star + NS. In M101, the spectrum is suggestive of a WR star and the accretor might be a BH (Liu2013).

X-ray binaries have lower luminosities though. Neither RLOF nor wind BHL can reach these levels. RLOF, ~LMXB, limited by the transverse section of the channel @ L1 while fast winds (~HMXB) mean low capture cross-sections. The first is essentially limited by the small sound speed compared to the orbital speed while the second is limited by the large wind speed with respect to the orbital speed. => wind-RLOF configuration leading to enhanced accretion, best of both worlds when the stellar mass loss rate is large. \cite{King2002} suggested that above a mass ratio unity, unstable RLOF mass transfer occurs on a thermal time scale and lead to high mass accretion rate onto the compact companion. However, \cite{Pavlovskii2017} recently found that RLOF mass transfer can be stable for large q, up to 7.5. Not high enough though to match the mass ratio in P13 where the stellar mass is at least 10 times larger than the mass of the accreting \ns.

% ------------------------------------------------
\section{Line-driven wind acceleration in SgXB}
\label{sec:}
% ------------------------------------------------

\begin{figure}
\begin{subfigure}{.5\textwidth}
\centering
\includegraphics[width=0.99\columnwidth]{Pictures/vel_prof.pdf}
  \label{fig:vel_prof}
\end{subfigure}
\phantom{p}\\
\begin{subfigure}{.5\textwidth}
\centering
\includegraphics[width=0.99\columnwidth]{Pictures/3D.png}
  \label{fig:3D}
\end{subfigure}
\caption{(upper panel) Wind velocity profiles of a representative B0.5 Ib supergiant star, HD 77581 \citep[the donor star in Vela X-1][]{Hiltner1972,Forman1973}. The green solid line is the $\beta$-velocity profile deduced by \cite{Gimenez-Garcia2016} from observations, while the green shaded region shows the uncertainties on the terminal wind speed. \cite{Sander2017} computed the hydrodynamic atmosphere solution for the wind stratification (red solid line), here fitted by a $\beta$-velocity profile (dashed red line). (lower panel) Illustration of the integration of the streamlines (orange) from the stellar surface (blue) to the Roche lobe of the accretor (transparent green).}
\label{fig:setup}
\end{figure} 

% - - - - - - - - - - - - - - - - - - - - - - - - 
\subsection{Empirical proxy to deduce acceleration from beta-laws}
\label{sec:}
% - - - - - - - - - - - - - - - - - - - - - - - - 

% - - - - - - - - - - - - - - - - - - - - - - - - 
\subsection{The equation of motion}
\label{sec:}
% - - - - - - - - - - - - - - - - - - - - - - - - 

% ------------------------------------------------
\section{Mass transferred via wind-RLOF}
\label{sec:}
% ------------------------------------------------

We differentiate 3 mass rates : $\dot{M}$, $\dot{M}_{\text{\mystar}}$ and $\dot{M}_{acc}$

in the sense of “entering the effective region of accretion" (either Roche lobe of the accretor or set by the accretion radius). Upper limit. For fast wind, effective cross-section set by accretion radius which decreases quickly and much below the radius of the Roche lobe of the accretor when the speed of the wind entering the Roche lobe gets larger than the orbital speed.

% - - - - - - - - - - - - - - - - - - - - - - - - 
\subsection{Fraction of stellar wind available for accretion}
\label{sec:}
% - - - - - - - - - - - - - - - - - - - - - - - - 

\begin{figure*}[!b]
\centering
\includegraphics[width=2\columnwidth]{Pictures/mdot_grid.png}
\caption{Logarithmic color maps of the fraction of stellar wind captured by the accretor as a function of the stellar filling factor and of the ratio of the terminal wind speed by the orbital speed. From left to right, the $\beta$ exponent is 1, 2 and 3, which means a more progressive acceleration up to the terminal speed. The first (resp. second) row stands for a mass ratio of 2 (resp. 15) which means, for a fixed 20 solar-masses supergiant donor, an accreting 10 solar-masses \bh (resp. a 1.3 solar-masses \ns).}
\label{fig:mdot}
\end{figure*} 

Mapping of the stellar surface feeding the accretor Roche lobe : contribution of the high latitudes (Fig.2 : the Mollweide projection)

\% of stellar mass loss rate entering the accretor Roche lobe (Fig.3)
1st row is M101? Or Cyg X-1?
2nd row is P13?

A comparison to BHL formula (Fig.4)

% - - - - - - - - - - - - - - - - - - - - - - - - 
\subsection{Accretion luminosity}
\label{sec:}
% - - - - - - - - - - - - - - - - - - - - - - - - 

\begin{center}
\begin{table}[!h]
\caption{Scaled X-ray luminosity of a classic \sgx (Vela X-1) and of a \ulx (P13) assuming a similar fraction of the wind captured of $\sim$ 5\% obtained with $q=15$, $f=95\%$, $\beta=2$ and $\eta=2$.}
\label{tab:params}
\centering
\begin{tabularx}{0.83\columnwidth}{c|c|c}
   & Vela X-1 & P13 \\
  \hline
  $\dot{M}/\dot{M}_{\text{\mystar}}$ & \multicolumn{2}{c}{$\sim$ 5\%} \\
  \hline
  $\dot{M}_{acc}/\dot{M}$ & 4\%  & 40\% \\
  $\zeta=L_X/\dot{M}_{acc}c^2$ & 10\% & 10\% \\
  $\dot{M}_{\text{\mystar}}$ & 5$\cdot$10$^{-7}$M$_{\odot}\cdot$yr$^{-1}$ & 10$^{-4}$M$_{\odot}\cdot$yr$^{-1}$ \\
  $L_X$ & 5$\cdot$10$^{36}$erg$\cdot$s$^{-1}$ & 10$^{40}$erg$\cdot$s$^{-1}$ \\
\end{tabularx}
\end{table}
\end{center}

Absolute values, how realistic? Indeed, capturing a large fraction of the wind from the donor star is a necessary condition to reach significant mass transfer rates, but it is not a guarantee. For instance, Vela X-1 : ...

Mass loss rates from \cite{Vink2000,Vink2001} (Sanders, private communication).


% ------------------------------------------------
\section{Discussion and conclusions}
\label{sec:}
% ------------------------------------------------

\begin{figure}
\centering
\includegraphics[width=0.99\columnwidth]{Pictures/BHL.png}
\caption{XXX}
\label{fig:BHL}
\end{figure}

BHL

RLOF

AGB and RSG donor stars have unknown wind launching process, no report of beta-law velocity profile but low terminal speeds and large mass loss rates => could also work for them (Heida).

\begin{acknowledgements}
IEM is grateful to Marianne Heida, Selma De Mink and Philipp Podsiadlowski for insightful exchanges on the possible scenarios leading to ultra-luminous X-ray sources. IEM has received funding from the Research Foundation Flanders (FWO) and the European Union's Horizon 2020 research and innovation program under the Marie Sk\l odowska-Curie grant agreement No 665501. IEM and JOS thank the Instituto de F\'{i}sica de Cantabria for its hospitality and for sponsoring a meeting which brought together the massive stars and X-ray binaries communities.
% THANK REFERREE
\end{acknowledgements}

%-------------------------------------------------------------------
XXX

\bibliographystyle{aa} %agsm}
\begin{tiny}
\bibliography{/Users/Ileyk/Documents/Bibtex/article_ULX}
\end{tiny}

\end{document}